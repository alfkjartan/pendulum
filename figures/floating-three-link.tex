\documentclass[tikz]{standalone}
\usetikzlibrary{calc,patterns}
\begin{document}

\newcommand{\nvar}[2]{%
    \newlength{#1}
    \setlength{#1}{#2}
}

% Define a few constants for drawing
\nvar{\dg}{0.3cm}
\def\dw{0.25}\def\dh{0.5}
\nvar{\ddx}{1.5cm}

% Define commands for links, joints and such
\def\centerofmass{%
  \fill[radius=3pt] (0,0) -- ++(3pt,0) arc [start angle=0,end angle=90] -- ++(0,-6pt) arc [start angle=270, end angle=180];%
  \draw[black, thick] (0,0) circle (3pt);%
}%

\def\link{\draw [double distance=1.5mm, very thick] (0,0)--}
\def\joint{%
    \filldraw [fill=white] (0,0) circle (5pt);
    \fill[black] circle (2pt);
}
\def\joint{%
    \filldraw [fill=white] (0,0) circle (5pt);
    \fill[black] circle (2pt);
}
\def\grip{%
    \draw[ultra thick](0cm,\dg)--(0cm,-\dg);
    \fill (0cm, 0.5\dg)+(0cm,1.5pt) -- +(0.6\dg,0cm) -- +(0pt,-1.5pt);
    \fill (0cm, -0.5\dg)+(0cm,1.5pt) -- +(0.6\dg,0cm) -- +(0pt,-1.5pt);
}
\def\robotbase{%
    \draw[rounded corners=8pt] (-\dw,-\dh)-- (-\dw, 0) --
        (0,\dh)--(\dw,0)--(\dw,-\dh);
    \draw (-0.5,-\dh)-- (0.5,-\dh);
    \fill[pattern=north east lines] (-0.5,-1) rectangle (0.5,-\dh);
}

% Draw an angle annotation
% Input:
%   #1 Angle
%   #2 Label
% Example:
%   \angann{30}{$\theta_1$}
\newcommand{\angann}[2]{%
    \begin{scope}[]
    \draw [dashed,] (0,0) -- (1.2\ddx,0pt);
    \draw [->, shorten >=3.5pt] (\ddx,0pt) arc (0:#1:\ddx);
    % Unfortunately automatic node placement on an arc is not supported yet.
    % We therefore have to compute an appropriate coordinate ourselves.
    \node at (#1/2-2:\ddx+8pt) {#2};
    \end{scope}
}

% Draw line annotation
% Input:
%   #1 Line offset (optional)
%   #2 Line angle
%   #3 Line length
%   #5 Line label
% Example:
%   \lineann[1]{30}{2}{$L_1$}
\newcommand{\lineann}[4][0.5]{%
    \begin{scope}[rotate=#2, inner sep=2pt]
        \draw[dashed, ] (0,0) -- +(0,#1)
            node [coordinate, near end] (a) {};
        \draw[dashed, ] (#3,0) -- +(0,#1)
            node [coordinate, near end] (b) {};
        \draw[|<->|] (a) -- node[fill=white] {#4} (b);
    \end{scope}
}

% Define the kinematic parameters of the three link manipulator.
\def\thetaone{-30}
\def\Lone{3}
\def\alphaOne{0.6}

\def\thetatwo{-30}
\def\Ltwo{2}
\def\alphaTwo{0.6}

\def\thetathree{40}
\def\Lthree{3}
\def\alphaThree{0.5}

\begin{tikzpicture}
  % coordinate system
  \begin{scope}[xshift=4cm, yshift=-1.5cm]
    \node[coordinate] (origin) at (0,0) {};
    \draw[->, thick] (origin) -- (1.5, 0) node [at end, below] {$x$};
    \draw[->,thick] (origin) -- (0,1.5) node [at end,left] {$z$};
  \end{scope}

  %robotbase is trunk segment (middle segment)
  \begin{scope}[rotate=90]
    \lineann[-0.7]{\thetaone}{\Lone}{$l_0$}
    \link(\thetaone:\Lone);
    \joint
    \pgfmathsetmacro{\dOne}{\alphaOne*\Lone}
    \begin{scope}[shift=(\thetaone:\dOne)]
      \node[coordinate] (CoM1) at (0,0) {};
      \angann{\thetaone}{$q_0$}
    \end{scope}
    \node[coordinate] (joint1) {};
    % Link 2
    \begin{scope}[shift=(\thetaone:\Lone), rotate=\thetaone]
        \angann{\thetatwo}{$q_1$}
        \lineann[-1.5]{\thetatwo}{\Ltwo}{$l_1$}
        \link(\thetatwo:\Ltwo);
        \joint
        \pgfmathsetmacro{\dTwo}{\alphaTwo*\Ltwo}
        \begin{scope}[shift=(\thetatwo:\dTwo)]
          \node[coordinate] (CoM2) at (0,0) {};
        \end{scope}
    \end{scope}
    \begin{scope}[shift=(\thetatwo:\Ltwo), rotate=\thetatwo]
      %  \grip
    \end{scope}

    % Link 3
    \begin{scope}[rotate=\thetaone]
    \begin{scope}[rotate=180]
        \angann{\thetathree}{$-q_2$}
        \lineann[1]{\thetathree}{\Lthree}{$l_2$}
        \link(\thetathree:\Lthree);
        \joint
        \pgfmathsetmacro{\dThree}{\alphaThree*\Lthree}
        \begin{scope}[shift=(\thetathree:\dThree)]
          \node[coordinate] (CoM3) at (0,0) {};
        \end{scope}
    \end{scope}
    \end{scope}
  \end{scope}

  \draw[->, thin] (origin) -- node[right] {$[q_3, q_4]$} (CoM1); 

    \begin{scope}[shift=(CoM1)]
      \centerofmass
    \end{scope}
    \begin{scope}[shift=(CoM2)]
      \centerofmass
    \end{scope}
    \begin{scope}[shift=(CoM3)]
      \centerofmass
    \end{scope}
    
\end{tikzpicture}

\end{document}
